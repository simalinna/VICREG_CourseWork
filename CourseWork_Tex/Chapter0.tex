\section{Введение}
\label{sec:Chapter0} \index{Chapter0}

% \todo[inline]
% {Введение. Здесь дается описание предметной области, обосновывается актуальность работы и в заключение ставится цель работы.    

%    Сделайте Ваше введение более развернутым: сейчас в нем есть краткое описание предметной области, но актуальность практически никак не подчеркивается - добавьте конкретные примеры, для каких задач и как сильно был уменьшен размер представления данных по сравнению с исходным пространством.  
% Цель работы - общее направление, в котором Вы хотите работать. Оно должно следовать из описания предметной области и актуальности. В Вашем случае - это исследование методов извлечения признаков.}

% описание предметной области
% актуальность работы
% цель работы

В настоящее время нейронные сети и другие алгоритмы машинного обучения используются повсеместно для огромного спектра различных задач, таких как распознавание образов и изображений, анализ текста и естественного языка, прогнозирование временных рядов, рекомендательные системы и т.д. С резким ростом размера доступных наборов данных как с точки зрения количества выборок, так и количества признаков в каждой выборке, важной задачей становится уменьшение размерности данных и сохранение числа признаков как можно более низким. Данная задача в машинном обучении называется "Обучение представлению данных". 

Зачастую, если данные содержат большое количество признаков, часть этих признаков являются шумовыми, то есть не содержат в себе полезной информации, а часть избыточными, то есть коррелируют с другими признаками и дублируют уже известную информацию. Если не избавляться от таких признаков, можно столкнуться со следующим рядом проблем:
\begin{itemize}
    \item переобучение;
    \item плохая интерпретируемость модели;
    \item невозможность визуализация данных;
    \item большая нагрузка на вычислительные ресурсы.
\end{itemize} 

Обучение представлению данных является действительно актуальной и важной задачей. Оно позволяет не только ускорить процесс обучения, но и повысить точность работы алгоритмов. Методы обучения представлению данных активно используются в различных прикладных сферах машинного обучения. Одной из распространенных сфер является задача обработки изображений. Для выделения признаков из изображений методы как правило обучаются и тестируются на наборе ImageNet \cite{ImageNet}. 

Рассмотрим насколько полезным может быть обучение представлению данных на примере задачи классификации изображений из данного набора. Он состоит из изображений разного размера, поэтому перед использованием их необходимо привести к единому формату. Популярные сверточные сети, используемые для решения данной задачи, такие как VGG \cite{VGG}, ResNet \cite{ResNet}, GoogleNet \cite{GoogleNet} и т.д., принимают на вход предобработанные изображения размером $224\times224$. То есть обрабатывают около 50.000 признаков. Современные методы обучения представлению данных позволяют сократить такой объем до вектора размерностью 2048 \cite{Vicreg}. Таким образом количество признаков уменьшается почти в 25 раз. 

Благодаря тому, что размер полученного представления в разы меньше, чем размер исходных данных, мы можем использовать более простой линейный классификатор для решения нашей задачи. Обучение будет производиться  гораздо быстрее, при этом качество его работы останется на том же уровне.

Методы обучения представлению данных можно разделить на две основные группы: методы, основанные на отборе признаков, и методы, основанные на извлечении признаков. Методы отбора признаков уменьшают размерность, выбирая подмножество признаков из исходного набора. Методы извлечения признаков преобразуют существующие признаки в новое пространство на основе линейных и нелинейных комбинаций из исходного набора.

Несмотря на то, что методы отбора признаков могут быть полезными, они имеют большой недостаток по сравнению с методами извлечения признаков: они не учитывают взаимодействия между признаками внутри выборки, что может привести к упущению значимых паттернов в данных. В целом, извлечение признаков обычно предпочтительнее отбора признаков, так как позволяет более гибко и эффективно использовать информацию из исходных данных.

Целью данной работы является исследование методов извлечения признаков. Для генерации нелинейных комбинаций они используют глубокие нейронные сети.


