\section{Заключение}
\label{sec:Chapter7} \index{Chapter7}

В рамках работы были исследованы методы обучения представлению данных. Был проведен обзор существующих подходов, на основе которого для дальнейших экспериментов был выбран метод извлечения признаков на основе обучения без учителя - VICReg.

Цель экспериментов заключалась в исследовании эффективности данного метода. Для исследования была реализована многомодульная программа, включающая в себя обучение нейронной сети с возможностью выбора архитектуры для получения компактных эмбедингов, а также тестирование, которое заключалось в обучении однослойного классификатора, принимающего на вход полученные эмбединги.  

В результате эксперимента удалось получить энкодер для генерации эмбедингов, который позволяет решать задачу классификации с более высокой точностью. Результаты были подтверждены на базе CIFAR-10. Энкодер, обученный с помощью метода VICReg повысил точность классификации на 21.5\% по сравнению с той же архитектурой, обученной с помощью обучения с учителем.

Полученный энкодер был успешно применен к базе Tiny ImageNet. При этом при дообучении на 10\% размеченных данных точность классификации достигла 86.4\%.

Проведенный анализ результатов 