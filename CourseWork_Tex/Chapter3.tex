\section{Постановка задачи}
\label{sec:Chapter3} \index{Chapter3}

В рамках задачи будем исследовать метод VICReg.

% \textbf{Исходные данные}

Рассмотрим два непересекающихся множества исходных данных. Обозначим их $A$ и $B$. Оба множества являются размеченными. Обозначим метки данных множеств как $M_A$ и $M_B$ для $A$ и $B$ соответственно. Множества меток $M_A$ и $M_B$ также являются непересекающимися. 

Разделим каждое из множеств $A$ и $B$ на два размеченных набора - тренировочный и тестовый. Обозначим данные наборы для множества $A$: $a_1$ и $a_2$, для множества $B$: $b_1$ и $b_2$, как тренировочный и тестовый набор соответственно.

Также введем множество неразмеченных данных $a_1^n$, которое будет представлять собой тренировочный набор множества A без меток: $a_1 \setminus M_A$.

Множество $a_1^n$ будем использовать для обучения энкодера с помощью метода VICReg. 

Задача будет тестироваться для обработки изображений, поэтому в качестве архитектуры энкодера будем использовать сверточную нейронную сеть.

% \textbf{Цель}

Цель заключается в том, чтобы проанализировать качество обучения энкодера с помощью метода VICReg. Для этого предлагается проверить точность классификации эмбедингов, полученных на выходе из энкодера после обучения. 

% \textbf{Схема работы}

Опишем подробнее шаги, которые необходимо выполнить для данной задачи.

\begin{enumerate}

    \item Выбрать сверточную сеть для архитектуры энкодера.
    \item Обучить энкодер на множестве $a_1^n$.
    \item Провести тестирование на множествах $A$ и $B$:
        \begin{enumerate}
            \item С помощью обученного энкодера получить эбединги для наборов $a_1$ и $b_1$. Множество полученных эмбедингов обозначим $\alpha_1$ и $\beta_1$ для каждого набора соответственно.
            \item Обучить однослойный линейный классификатор на множествах $\alpha_1$ и $\beta_1$.
        \end{enumerate}    
    \item Оценить точность обученного классификатора на наборах $a_2$ и $b_2$:
        \begin{enumerate}
            \item Для оценки точности классификации будем использовать метрику Accuracy и проверять точность для Top-1 и Top-5 предсказаний.
        \end{enumerate}    
    \item В конце сравним:
        \begin{enumerate}
            \item Насколько различаются точности классификации набора $a_2$, если:
                \begin{itemize}
                    \item обучать энкодер на множестве $a_1$ с помощью обучения с учителем;
                    \item обучать энкодер на множестве $a_1^n$  с помощью метода VICReg.
                \end{itemize}
            \item Насколько различаются точности классификации наборов $a_2$ и $b_2$.
        \end{enumerate}
        
\end{enumerate}